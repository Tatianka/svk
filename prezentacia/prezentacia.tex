\documentclass[mathserif]{beamer}
\usepackage[utf8]{inputenc}
\usepackage[slovak]{babel}
\usepackage{amssymb}
\usepackage{amsmath}
\usepackage{amsthm}
\usepackage{dsfont}
\usepackage{mathrsfs}
\usepackage{amsfonts}
\usepackage{graphicx}
\usepackage{lmodern}
\usepackage[T1]{fontenc}
\usepackage{fontenc}
\usepackage{bm} %pisat boldom v math-mode
\usepackage{multirow}

\usetheme{Warsaw}
\usecolortheme{wolverine}
\setbeamercolor{block body alerted}{bg=yellow}

\def\R{{\cal R}} % znak pre regulárne jazyky
\def\L{\mathscr{L}} % zvyšné triedy jazykov
\def\P{{\cal P}} % potenčná množina
\def\N{\mathds{N}} %prirodzene cisla
\def\re{Regex}
\def\e{Eregex}
\def\le{LEregex}
\def\nle{nLEregex}
\def\rel{{\mathscr{L}_{RE}}}
\def\el{\mathscr{L}_{ERE}}
\def\lel{\mathscr{L}_{LERE}}
\def\nlel{\mathscr{L}_{nLERE}}
\def\lookahead{\text{(?=}}
\def\nlookahead{\text{(?!~}}
\def\lookbehind{\text{(?\textless =}}
\def\nlookbehind{\text{(?\textless !~}}

\title{Moderné regulárne výrazy}
\author[Tatiana Tóthová]{Tatiana Tóthová \\ \small školiteľ: RNDr. Michal Forišek PhD.}
\institute{Fakulta matematiky, fyziky a informatiky \\ Univerzita Komenského}
\date{22.apríla 2015}

\beamertemplatenavigationsymbolsempty 
\begin{document}

\begin{frame}
\titlepage
\end{frame}

\begin{frame}
\textbf{Motivácia:}
\begin{itemize}
\item teoretický model regulárnych výrazov bol v praxi rozšírený o nové konštrukcie
\end{itemize}

\vspace{20pt}
\textbf{Cieľ	:}
\begin{itemize}
\item preskúmať z hľadiska formálnych jazykov -- Chomského hierarchia, vlastnosti triedy jazykov, zložitosť
\end{itemize}
\end{frame}

\begin{frame}{Definícia}
Regex tvoria \textbf{znaky} a \textbf{metaznaky}

\vspace{10pt}
Základné operácie:
\begin{itemize}
\item $(e_1)(e_2)$ zreťazenie (bez znaku)
\item $(e_1) | (e_2)$ alternácia
\item $(e_1)*$ Kleeneho uzáver
\end{itemize}

Navyše: ľubovoľný znak . , začiatok slova $\textasciicircum$, koniec slova \$

\vspace{10pt}
Ďalšie operácie: 
\begin{itemize}
\item množiny znakov $[abc],[a-z] $
\item komplementy množín znakov $\left[\textasciicircum abc\right], \left[\textasciicircum a-z\right]$
\item opakovania $?, \lbrace n\rbrace, \lbrace n,m\rbrace, + $
\end{itemize}

\vspace{10pt}
\textbf{Regex má konečnú dĺžku!}


\end{frame}

\begin{frame}{Zložitejšie konštrukcie (1)}
\textbf{Číslovanie zátvoriek} zľava doprava, podľa otváracej zátvorky. Nečíslujú sa konštrukcie (?\dots)

\vspace{20pt}
\textbf{Spätné referencie $\backslash k$} -- odkazujú sa na $k$-te zátvorky
$\displaystyle \alpha~ \mathop(_k ~\beta ~\mathop)_k ~\gamma ~\backslash k ~\delta$

$$w = \underbrace{x_1\dots x_{i-1}}_\alpha 
 \overbrace{\underbrace{x_i\dots x_{j-1}}_{ \displaystyle{\mathop(_k\beta \mathop)_k}}}^{w_k} 
 \underbrace{x_j\dots x_{l-1}}_\gamma 
 \overbrace{\underbrace{x_l\dots x_{m-1}}_{\backslash k}}^{w_k}
 \underbrace{x_{m}\dots x_{n}}_\delta$$
\vspace{10pt} 
$$ w_k = x_i\dots x_{j-1} = x_l\dots x_{m-1} $$

\end{frame}
\begin{frame}{Zložitejšie konštrukcie (2)}
\textbf{Lookahead} $\lookahead\dots)$ -- nazeranie dopredu:
$\alpha(?=\beta)\gamma$ $$w = \underbrace{x_1\dots x_{i-1}}_\alpha \underbrace{\overbrace{x_i \dots x_j}^\beta x_{j+1} \dots x_n }_\gamma$$ 

\textbf{Lookbehind} $\lookbehind\dots)$ -- nazeranie dozadu: $\alpha(?<=\beta)\gamma$

$$w = \underbrace{x_1\dots x_{i-1} \overbrace{x_i \dots x_j}^\beta}_\alpha \underbrace{x_{j+1} \dots x_n }_\gamma$$

Ich \textbf{negatívne formy} $\nlookahead\dots),\nlookbehind\dots)$ $\rightarrow$ opačná akceptácia (slová, ktoré sa v danom jazyku nenachádzajú)
\end{frame}

\begin{frame}{Triedy}
\begin{description}
\item[Regex] základné operácie a znaky . \textasciicircum ~\$ ( = $\R$)
\item[Eregex] + spätné referencie
\item[LEregex] + lookahead, lookbehind
\item[nLEregex] + negatívny lookahead, negatívny lookbehind
\end{description}
\vspace{30pt}
Triedy jazykov: $\R,~\el,~\lel,~\nlel$
\end{frame}

\begin{frame}{Hierarchia}
\begin{block}{Veta}
Trieda nad $\re$ s pozitívnym a negatívnym lookaroundom je $\R$.
\end{block}

\vspace{15pt}

\begin{block}{Hierarchia}
$\R \subsetneq \el \subsetneq \lel \subseteq \nlel \subsetneq \L_{CS}$
\end{block}
\begin{block}{}
$\lel$ a $\nlel$ sú neporovnateľné s $\L_{CF}$
\end{block}

\end{frame}

\begin{frame}{Pumpovacia lema}
\begin{block}{Veta}
Jazyk všetkých platných výpočtov Turingovho stroja patrí do $\lel$.
\end{block}

\begin{block}{Veta}
Nech $\alpha\in\le$ nad unárnou abecedou $\Sigma = \lbrace a \rbrace$ taký, že neobsahuje lookahead s $\mathdollar$ ani lookbehind s $\textasciicircum$ vnútri iterácie ($*$). Existuje konštanta $N$ taká, že ak $w \in L(\alpha)$ a $\vert w \vert > N$, potom existuje dekompozícia $w=xy$ s nasledujúcimi vlastnosťami:
\begin{enumerate}[(i)]
\item $\vert y \vert \geq 1$
\item $\exists k \in \N,~k\neq 0;~\forall j = 1,2,\ldots: xy^{kj} \in L(\alpha)$
\end{enumerate}
\end{block}
\end{frame}

\begin{frame}{Uzavretosť na zreťazenie}
\begin{block}{Veta}
$\lel$ je uzavretá na zreťazenie.
\end{block}
\textbf{Dôkaz.} Nech $\alpha,\beta\in\le$. Regex pre $L(\alpha)L(\beta)$ je
$$
 \lookahead \mathop(_1 \alpha \mathop)_1 \mathop(_{k+2}\beta \mathop)_{k+2} \mathdollar) ~\alpha ' \backslash k\text{+2}~ \lookbehind ~\textasciicircum\backslash 1~\beta ')
$$
\end{frame}

\begin{frame}{Nový formalizmus}
\begin{itemize}
\item konfigurácia: $(r_1\dots \lceil r_i\dots r_n,w_1 \dots \lceil w_j \dots w_m)$
\item indexovateľné zátvorky
\item krok výpočtu 
\\ \vspace{5pt} $\displaystyle{(r_1 \dots \mathop{(}_k \dots \mathop{)}_k\lceil * \dots r_n, w_1 \dots \mathop{w_a}^k \dots \mathop{w_b}^{k'} \dots \lceil w_j \dots w_m)}$
$$(1) ~\vdash(r_1 \dots \mathop{(}_k \dots \mathop{)}_k *\lceil \dots r_n, w_1 \dots \mathop{w_a}^k \dots \mathop{w_b}^{k'} \dots \lceil w_j \dots w_m)$$
$$(2) ~\vdash(r_1 \dots \mathop{(}_k\lceil \dots \mathop{)}_k * \dots r_n, w_1 \dots w_a \dots w_b \dots \lceil \mathop{w_j}^k \dots w_m)$$
\item akceptačný výpočet $(\lceil r, \lceil w)\vdash^* (r \lceil, w \lceil)$
\item jazyk
\end{itemize}
\end{frame}

\begin{frame}{Dĺžka akceptačného výpočtu}
\begin{block}{Lema}
Nech $\alpha \in \e$ a $w \in L(\alpha)$. Potom existuje akceptačný výpočet, ktorý má najviac $5\cdot|\alpha|\cdot|w|$ konfigurácií.
\end{block}

\vspace{20pt}
Počet všetkých možných konfigurácií $\alpha\in\le$ je $$O(|\alpha|\cdot|w|^{|\alpha|+2})$$
Kvôli vlastnostiam lookaroundu lepší odhad nemáme.
\end{frame}

\begin{frame}{Priestorová zložitosť}
\begin{block}{Veta.}
$\lel \subseteq NSPACE(\log n)$
\end{block}
\textbf{Dôkaz.} Na páske si pamätáme informáciu z poschodových symbolov -- vo forme adries

\vspace{10pt}
Zo Savitchovej vety vyplýva: $\lel \subseteq DSPACE(\log^2 n)$
\begin{block}{Veta}
$\nlel \subseteq DSPACE(\log^2 n)$
\end{block}
\textbf{Dôkaz.} Idea Savitchovej vety, ale s konfiguráciami z formalizmu. Pre každý negatívny lookaround spúšťame nový Turingov stroj s opačnou akceptáciou.
\end{frame}

\begin{frame}{Regex aj jazyk na vstupe}
$r=|regex|,~w=|word|$
\begin{block}{Veta}
$L(regex\#word) \in NSPACE(r\log w)$, regex $\in \le$
\end{block}
\textbf{Dôkaz.} Vyplýva z $\lel \subseteq NSPACE(\log n)$, dĺžka regexu už viac nie je konštanta.
\begin{block}{Veta}
Nech počet konfigurácií v akceptačnom výpočte je polynomiálny od $r$ a $w$, potom
$L(regex\#word) \in DSPACE(r\log^2 w)$, regex~$\in \nobreak\le$
\end{block}
\textbf{Dôkaz.} Idea Savitchovej vety, ale s konfiguráciami z formalizmu.
\end{frame}

\begin{frame}{Na záver}
\begin{alertblock}{}
\begin{center}
\vspace{10pt}
\large{Ďakujem za pozornosť!}
\vspace{10pt}
\end{center}
\end{alertblock}
\end{frame}

\begin{frame}{Problém s lemou 2}
\begin{block}{Lema 2}
Nech $\alpha \in \lel,~ s \in L(\alpha),~ r = |\alpha|$ a $w = |s|$. Potom existuje akceptačný výpočet, ktorý má nanajvýš $O(r^2w^3)$ konfigurácií.
\end{block}

\vspace{10pt}
Pripomeňme lemu 1
\begin{block}{Lema 1}
Nech $\alpha \in \el$ a $w \in L(\alpha)$. Potom existuje akceptačný výpočet, ktorý má najviac $5\cdot|\alpha|\cdot|w|$ konfigurácií.
\end{block}
Doplnenie na $\lel$? Funguje len pre regexy s konštantnou hĺbkou vnorenia lookaroundov.

\vspace{5pt}
Počet všetkých možných konfigurácií je $O(|\alpha|\cdot|w|^{|\alpha|+2})$
\end{frame}

\begin{frame}{Komplikovanosť lookaheadu s \$ a lookbehindu s \textasciicircum ~ vnútri $*$}
Napríklad regex
$$(\lookahead \underbrace{
 \underbrace{ \lookahead (a^m)*\$ )}_{a^{km},~ k\in\N} 
 \underbrace{(a^{m+1})* a\lbrace 1,m-1\rbrace \$ ~|~ a^m \$ }_{\text{vie } a* \text{ okrem } a^{m+1},~a^{m(m+1)l},~ l\in\N} 
 }_{a^{km} \text{ také, že nevie } a^{m(m+1)l}, ~l\in\N }
  ) a^m)+$$
generuje konečný jazyk obsahujúci slová 
\begin{itemize}
\item $a^m$
\item $a^{2m}$
\item \dots
\item $a^{(m-1)(m+1)}$
\end{itemize} 
Hlavný lookahead je spúšťaný každú iteráciu, teda pre slovo $a^{zm}$ musí matchovať všetky $a^{im}$ pre $i\in\lbrace 1,\dots,z\rbrace$.
\end{frame}

\end{document}
