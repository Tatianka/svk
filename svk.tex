% !Mode:: "TeX::UTF-8"
\documentclass{svk_long_sk}
%% ak pisete po anglicky, pouzijete namiesto horneho riadku
%% \documentclass{svk_long_en}
\usepackage{mathrsfs}
\usepackage{amstext}
\usepackage{amssymb}
\usepackage{dsfont}
\usepackage[shortlabels]{enumitem}
\usepackage{color}
\usepackage{amsmath}

\definecolor{ltyelloow}{rgb}{0.99,1,0.96}

%definície pre foju
%\newtheorem{veta}{Veta}[section] 
%\newtheorem{df}[veta]{Definícia}
%\newtheorem{lema}[veta]{Lema}
%\newtheorem{dosledok}[veta]{Dôsledok}
\def\R{{\cal R}} % znak pre regulárne jazyky
\def\L{\mathscr{L}} % zvyšné triedy jazykov
\def\P{{\cal P}} % potenčná množina
\def\N{\mathds{N}} %prirodzene cisla
\def\re{Regex}
\def\e{Eregex}
\def\le{LEregex}
\def\nle{nLEregex}
\def\rel{{\mathscr{L}_{RE}}}
\def\el{\mathscr{L}_{ERE}}
\def\lel{\mathscr{L}_{LERE}}
\def\nlel{\mathscr{L}_{nLERE}}
\def\lookahead{\text{(?=}}
\def\nlookahead{\text{(?!~}}
\def\lookbehind{\text{(?\textless =}}
\def\nlookbehind{\text{(?\textless !~}}

\def\todo{{\color{red} TODO!!!}}
\def\TODO{{\color{red} TODO!!!}}

\begin{document}
\title{Moderné regulárne výrazy}

\author{Tatiana Tóthová
\email{tothova166@uniba.sk}}
%% vsimnite si, ze u autorov sa nepisu tituly
%% prikaz \inst sluzi ako odkaz do zoznamu institucii
%% (vid. nizsie)

%% skolitela nepiste medzi autorov, ale v tejto casti
%% ak praca nema skolitela, jednoducho vynechajte
\supervisor{Michal Fori\v{s}ek
\email{forisek@dcs.fmph.uniba.sk}}

%% nasleduje kratka verzia nazvu clanku a 
%% zoznam autorov (bez krstnych mien)
%% tieto informacie sa zobrazuju v hlavicke
\titlerunning{Moderné regulárne výrazy}
\authorrunning{Tóthová}

\institute{
Katedra informatiky,
FMFI UK,
Mlynská Dolina
842~48~Bratislava}

\maketitle

\begin{abstract}
Regulárne výrazy implementované v súčasných programovacích jazykoch ponúkajú omnoho viac operácií ako pôvodný model z teórie jazykov. Už konštrukciou spätných referencií bola prekročená hranica regulárnych jazykov. Náš model obsahuje naviac konštrukcie lookahead a lookbehind. V článku uvedieme zaradenie modelu zodpovedajúcej triedy jazykov do Chomského hierarchie, vlasnosti tejto triedy a výsledky z oblasti priestorovej zložitosti.

\keywords{regulárny výraz, regex, lookahead, lookbehind, spätné referencie}
\end{abstract}

\section{Úvod}
Regulárne výrazy vznikli v 60tych rokoch v teórii jazykov ako ďalší model na vyjadrenie regulárnych jazykov. Z takéhoto popisu ľudský mozog rýchlejšie pochopil o aký jazyk sa jedná, než zo zápisu konečného automatu, či regulárnej gramatiky. Ďalšou výhodou bol kratší a kompaktný zápis.

Vďaka týmto vlastnostiam boli implementované ako vyhľadávací nástroj. Postupom času sa iniciatívou používateľov s vyššími nárokmi pridávali nové konštrukcie na uľahčenie práce. Nástroj takto rozvíjali až do dnešnej podoby. My sa budeme opierať o špecifikáciu regulárnych výrazov v jazyku Python \cite{Python3Documentation}.

Ako čoskoro zistíme, nové regulárne výrazy vedia reprezentovať zložitejšie jazyky ako regulárne, preto je dobré ich nejako odlíšiť. V literatúre sa zaužíval výraz ,,regex'' z anglického \textit{regular expression}, ktorý budeme používať aj my.

\subsection{Základná definícia}

Regulárne výrazy sú zložené zo znakov a metaznakov. Znak $a$ predstavuje jazyk $L(a)=\lbrace a \rbrace$. Metaznak alebo skupina metaznakov určuje, aká operácia sa so znakmi udeje. Základné operácie sú zreťazenie (je definované tým, že regulárne výrazy idú po sebe, bez metaznaku), Kleeneho uzáver ($(0-\infty)$-krát zopakuj výraz, metaznak $*$) a alternácia (vyber výraz naľavo alebo napravo, metaznak~$|$~). Naviac sa využíva metaznak $\backslash$, ktorý robí z metaznakov obyčajné znaky a okrúhle zátvorky na logické oddelenie regulárnych výrazov.

Pre regulárny výraz $\alpha$ a slovo $w\in L(\alpha)$ hovoríme, že $\alpha$ vyhovuje slovu $w$ resp. $\alpha$ matchuje slovo $w$. Tiež budeme hovoriť, že $\alpha$ generuje jazyk $L(\alpha)$.


\subsection{Nové jednoduché konštrukcie}
\begin{itemize}
\item $+$ -- Kleeneho uzáver opakujúci $(1-\infty)$-krát
\item $\lbrace n,m \rbrace$ ($\lbrace n \rbrace$) -- opakuj regulárny výraz aspoň $n$ a najviac $m$-krát (opakuj $n$-krát)
\item $[a_1a_2\dots a_n]$ -- predstavuje ľubovoľný znak z množiny $\lbrace a_1,\dots,a_n \rbrace$
\item $[\textasciicircum a_1a_2\dots a_n]$ -- predstavuje ľubovoľný znak, ktorý nepatrí do množiny $\lbrace a_1,\dots,a_n \rbrace$
\item . -- predstavuje ľubovoľný znak
\item ? -- ak samostatne: opakuj 0 alebo 1-krát \\
ak za operáciou: namiesto greedy implementácie použi minimalistickú, t.j. zober čo najmenej znakov (platí pre $*,+,?,\lbrace n,m \rbrace$)\footnote{všetky spomenuté operácie sú implementované greedy algoritmom}
\item $\textasciicircum$ -- metaznak označujúci začiatok slova
\item \$ -- metaznak označujúci koniec slova
\item (?\# komentár) -- komentár sa pri vykonávaní regexu úplne ignoruje
\end{itemize}

Všetky tieto konštrukcie sú len ,,kozmetickou'' úpravou pôvodných regexov -- to isté vieme popísať pôvodnými regulárnymi výrazmi, akurát je to dlhšie a menej prehľadné

Rozdiely medzi minimalistickou a greedy verziou operácií vníma iba používateľ, pretože ak existuje zhoda regexu so slovom, v oboch prípadoch sa nájde. Viditeľné sú až pri výstupnej informácii pre používateľa, ktorú môže použiť ďalej.

\subsection{Zložitejšie konštrukcie}
\subsubsection*{Spätné referencie}
Najprv potrebujeme očíslovať všetky zátvorky v regexe. Číslujú sa všetky, ktoré nie sú tvaru $(?\dots)$. Poradie je určnené podľa otváracej zátvorky.

Spätné referencie umožňujú odkazovať sa na konkrétne zátvorky. Zápis je $\backslash k$ a môže sa nachádzať až za $k$-tymi zátvorkami. Skutočná hodnota $\backslash k$ sa určí až počas výpočtu -- predstavuje posledné podslovo zo vstupu, ktoré matchovali $k$-te zátvorky.

\subsubsection*{Lookahead}
Zapísaný formou $\lookahead \dots)$, vnútri je validný regex.

Keď v regexe prídeme na pozíciu lookaheadu, zoberieme regex vo vnútri. V slove sa snažíme matchovať ľubovoľný prefix zostávajúcej časti slova. Ak sa to podarí, pokračujeme v regexe ďalej a v slove od pozície, kde lookahead začínal (tzn. akokeby v regexe nikdy nebol).

Má aj negatívnu verziu -- negatívny lookahead $\nlookahead \dots)$. Vykonáva sa rovnako ako lookahead, ale má otočnú akceptáciu. Teda ak neexistuje prefix, ktorý by vedel matchovať, akceptuje.

\subsubsection*{Lookbehind}
Zapísaný formou $\lookbehind \dots)$, vnútri je validný regex.

Pri výpočte zoberieme regex vnútri lookbehindu a snažíme sa vyhovieť ľubovoľnému sufixu už matchovanej časti slova. Ak vyhovieme, pokračujeme v slove a regexe akoby tam lookbehind vôbec nebol.

Aj lookbehind má negatívnu verziu -- negatívny lookbehind $\nlookbehind \dots)$ -- a pracuje analogicky ako negatívny lookahead.
\\ \\ Lookahead a lookbehind (spolu nazývané jedným slovom lookaround) sú v rôznych implementáciách rôzne obmedzované, aby výpočet netrval príliš dlho. V teórii tieto obmedzenia ignorujeme a prezentujeme model v plnej sile -- výsledky tak prezentujú hornú hranicu toho, čo implemetnácie dokážu. 

\subsection{Priorita}

Pri interakcii toľkých operácií je nutné vedieť ich priority. Existujú také, ktoré sa správajú ako znak, čomu zodpovedajú [],[\textasciicircum],. a každá spätná referencia. Špeciálne sú lookahead a lookbehind -- tie sa vykonajú hneď akonáhle na ne narazíme. Ostatné zoradíme v tabuľke: 

\begin{center}
\begin{tabular}{|c||c|c|c|c|}
\hline
priorita &3&2&1& 0  \\
\hline
operácia & () & $*$ + ? $\lbrace \rbrace$ & zreťazenie & | \\ \hline
\end{tabular}
\end{center}

\subsection{Triedy a množiny}
Kvôli porovnávaniu a vytvoreniu hierarchie sme rozdelili operácie do niekoľkých množín:
\begin{description}
\item[$\re$] -- množina operácií, pomocou ktorých vieme popísať iba regulárne jazyky; presnejšie všetky znaky a metaznaky (bez zložitejších operácií)
\item[$\e$] -- $\re$ so spätnými referenciami
\item[$\le$] -- $\e$ s pozitívnym lookaroundom
\item[$\nle$] -- $\le$ s negatívnym lookaroundom
\item[$\rel$] -- trieda jazykov nad $\re$ ($= \R$)
\item[$\el$] -- trieda jazykov nad $\e$
\item[$\lel$] -- trieda jazykov nad $\le$
\item[$\nlel$] -- trieda jazykov nad $\nle$
\end{description}

Trieda $\lel$ už bola hlbšie preskúmaná a výsledky čerpáme z článkov \cite{ExtendedRegexPower} a \cite{ExtendedRegexIntersec}.
 
\section{Formalizácia modelu}
Pri zložitejších dôkazoch sa ukázala potreba lepšieho formalizmu, než len množiny operácií. Kvôli jednoduchosti sme vybrali len potrebné operácie -- zreťazenie, alternáciu, Kleeneho $*$, spätné referencie a pozitívny a negatívny lookaround -- a pokúsili sa ho vyjadriť ako model, ktorý pracuje v krokoch podobne ako Turingov stroj.

Základným prvkom je \textbf{konfigurácia}. Je to dvojica regex $r_1 \dots r_n$ a vstupné slovo $w_1 \dots w_m$, pričom v oboch reťazcoch sa navyše nachádza ukazovateľ pozície $\lceil$ (ako hlava Turingovho stroja): $(r_1\dots \lceil r_i\dots r_n,w_1 \dots \lceil w_j \dots w_m)$. Špeciálne rozoznávame počiatočnú konfiguráciu $(\lceil r_1 \dots r_n,\lceil w_1 \dots w_m)$ a akceptačnú konfiguráciu $(r_1 \dots r_n \lceil, w_1 \dots w_m \lceil)$.

Znaky slova ($w_1\dots w_m$) v konfigurácii budú niesť nejakú informáciu naviac, preto použijeme poschodové symboly. Na najspodnejšom poschodí bude uchovaná informácia o skutočnom znaku na tej pozícii v slove a nebude sa meniť. Obsah vrchných poschodí bude špecifikovaný v kroku výpočtu. Z celého poschodového symbolu budeme zobrazovať vždy len tú informáciu, ktorú práve potrebujeme, tzn. $w_j$ bude predstavovať iba znak na najspodnejšom poschodí. Nezabúdajme však, že na ostatných poschodiach môže mať zapísané čokoľvek, čo možno neskôr v inom kroku využijeme.

Najprv si definujeme potrebné pojmy indexovateľnosti a alternovateľnosti. \textit{Indexovateľné zátvorky} sú také, kde za otváracou zátvorkou nenasleduje ? (t.j. všetky prípady okrem lookaroundu). Tieto zátvorky budeme číslovať. \textit{Alternovateľný regex} je taký, ktorý sa môže vyskytovať v alternácii. Sú 3 prípady: regex sa môže nachádzať naľavo od $|$, napravo od $|$ alebo je z oboch strán ohraničený $|$. Ak alternácia nie je uzavretá zátvorkami, ľavý a pravý krajný regex siaha až ku kraju slova, pretože alternácia je operácia s najmenšou prioritou. Inak sú pre nich hranicou zátvorky uzatváracie alternáciu.

Vďaka definovaniu týchto pojmov vidíme, že vieme algoritmicky zistiť, ktoré zátvorky sú indexovateľné a ktoré regexy sú alternovateľné.

\begin{definition}\label{dfkonfig}
\textbf{Krok výpočtu} je relácia $\vdash$ na konfiguráciách definovaná nasledovne:
%%%%%%%%%%%%%%%%%%%%%%%%%%%%%%%%%%%%%
\begin{enumerate}[I.]
\item \underline{zhoda písmenka}
\\ $\displaystyle \forall a \in \Sigma: ~(r_1 \dots \lceil a \dots r_n, w_1 \dots \lceil a \dots w_m)$ 
$$ \vdash (r_1 \dots a\lceil \dots r_n, w_1 \dots a \lceil \dots w_m)$$
\item \underline{zápis adresy zátvorky (}
\\ Nech ( je indexovateľná, $k$-ta v poradí: 
\\ $\displaystyle{ (r_1 \dots \lceil ( \dots r_n, w_1 \dots \lceil w_j \dots w_m)}$ 
$$ (1) \vdash (r_1 \dots (\lceil \dots r_n, w_1 \dots \lceil \mathop{w_j}^k \dots w_m)$$
Ak za jej uzatváracou zátvorkou nasleduje $*$, t.j. $\alpha$ je tvaru 
$\displaystyle r_1 \dots \lceil ( \dots ) * \dots r_n$, potom 
$$ (2) \vdash (r_1 \dots (\dots )* \lceil \dots r_n, w_1 \dots \lceil \mathop{\mathop{w_j}^k}^{k'} \dots w_m ) $$
\item \underline{zápis adresy zátvorky )}
\\ Nech ) je indexovateľná, $k$-ta v poradí: $\displaystyle (r_1 \dots \lceil ) \dots r_n, w_1 \dots \lceil w_j \dots w_m)$ 
$$ \vdash (r_1 \dots )\lceil \dots r_n, w_1 \dots \lceil \mathop{w_j}^{k'} \dots w_m)$$
\item \underline{výber možnosti v alternácii}
\\ Nech podslová $\alpha_1, \alpha_2,\dots,\alpha_A$ regexu $\alpha$ sú všetkými členmi zobrazenej alternácie: $(r_1 \dots \lceil \alpha_1 | \alpha_2 | \dots | \alpha_A \dots r_n, w_1\dots \lceil w_j \dots w_m)$
$$(1) \vdash \textit{ďalší prechod v }\alpha_1$$
$$(2) \vdash (r_1 \dots \alpha_1 | \lceil \alpha_2 | \dots | \alpha_A \dots r_n, w_1\dots \lceil w_j \dots w_m)$$
$$\vdots$$
$$(A) \vdash (r_1 \dots \alpha_1 | \alpha_2 | \dots | \lceil \alpha_A \dots r_n, w_1\dots \lceil w_j \dots w_m)$$
\item \underline{skok z dokončenej možnosti na koniec alternácie}
\\ Nech podslová $\alpha_1, \alpha_2,\dots,\alpha_A$ regexu $\alpha$ sú všetkými členmi zobrazenej alternácie, potom pre všetky možnosti:
\\ $(r_1 \dots \alpha_1 \lceil | \alpha_2 | \dots | \alpha_A \dots r_n, w_1\dots \lceil w_j \dots w_m)$,
\\ $(r_1 \dots \alpha_1 | \alpha_2 \lceil | \dots | \alpha_A \dots r_n, w_1\dots \lceil w_j \dots w_m)$,
\\ \vdots \\ $(r_1 \dots \alpha_1 | \alpha_2 | \dots | \alpha_{A-1} \lceil | \alpha_A \dots r_n, w_1\dots \lceil w_j \dots w_m)$
$$\vdash (r_1 \dots \alpha_1 | \alpha_2 | \dots | \alpha_A \lceil \dots r_n, w_1\dots \lceil w_j \dots w_m)$$
\item \underline{skok Kleeneho $*$ za znakom}
\\ $(r_1\dots a\lceil *\dots r_n, w_1\dots\lceil w_j\dots w_m)$
$$(1) \vdash (r_1\dots a *\lceil\dots r_n, w_1\dots\lceil w_j\dots w_m)$$
$$(2) \vdash (r_1\dots \lceil a *\dots r_n, w_1\dots\lceil w_j\dots w_m)$$
\item \underline{skok Kleeneho $*$ za regexom v ( )}
\\ $\displaystyle{(r_1 \dots \mathop{(}_k \dots \mathop{)}_k\lceil * \dots r_n, w_1 \dots \mathop{w_a}^k \dots \mathop{w_b}^{k'} \dots \lceil w_j \dots w_m)\footnotemark}$
\footnotetext{Podľa definície spätných referencií platí podsledné podslovo nájdené regexom v $k$-tych zátvorkách. Pri tejto pracovnej pozícii v regexe je zrejmé, že nejde o prvý prechod cez tieto zátvorky a teda existuje také $a,b$, že $k$ je v slove nad $w_a$ a $k'$ nad $w_b$. Ak nastane prechod (2), pôvodné horné indexy $k, k'$ miznú a pridáva sa $k$ nad $w_j$.}
$$(1) ~\vdash(r_1 \dots \mathop{(}_k \dots \mathop{)}_k *\lceil \dots r_n, w_1 \dots \mathop{w_a}^k \dots \mathop{w_b}^{k'} \dots \lceil w_j \dots w_m)$$
$$(2) ~\vdash(r_1 \dots \mathop{(}_k\lceil \dots \mathop{)}_k * \dots r_n, w_1 \dots w_a \dots w_b \dots \lceil \mathop{w_j}^k \dots w_m)$$
\item \underline{špeciálny ukazovateľ -- zjavenie}
\\ $\displaystyle{(r_1 \dots \lceil \backslash k \dots r_n,w_1\dots\mathop{w_a}^k\dots\mathop{w_b}^{k'}\dots \lceil w_j\dots w_m)}$
$$\vdash(r_1 \dots \lceil \backslash k \dots r_n,w_1\ldots\intercal\mathop{w_a}^k \dots \mathop{w_b}^{k'} \dots \lceil w_j\dots w_m)$$	
\item \underline{porovnávanie spätnej referencie}
\\ $\displaystyle{(r_1 \dots \lceil \backslash k \dots r_n,w_1\dots\mathop{w_a}^k\ldots \intercal w_c \dots \mathop{w_b}^{k'} \dots \lceil w_j\dots w_m)}$, kde $a\leq c < b$ a zároveň $w_c = w_j$\footnote{$w_c$ a $w_j$ môžu byť poschodové symboly, avšak pri tejto rovnosti poschodia ignorujeme -- chceme porovnať iba písmenká v slove, prislúchajúce týmto pozíciám.}
$$\vdash(r_1 \dots \lceil \backslash k \dots r_n,w_1\dots\mathop{w_a}^k\dots w_c\intercal \dots \mathop{w_b}^{k'} \dots  w_j\lceil\dots w_m)$$
\item \underline{špeciálny ukazovateľ -- zmiznutie}
\\ $\displaystyle{(r_1 \dots \lceil \backslash k \dots r_n,w_1\dots\mathop{w_a}^k\ldots \intercal  \mathop{w_b}^{k'} \dots \lceil w_j\dots w_m)}$
$$\vdash(r_1 \dots  \backslash k \lceil \dots r_n,w_1\dots\mathop{w_a}^k\dots \mathop{w_b}^{k'} \dots  w_j\lceil\dots w_m)$$
\item \underline{lookahead -- začiatok a jeho záznam} 
\\ Nech $\lookahead \dots$) je $k$-ty pozitívny lookahead v poradí: 
\\ $\displaystyle{(r_1\dots \lceil \lookahead \dots) \dots r_n, w_1\dots\lceil w_j\dots w_m )}$
$$ \vdash(r_1\dots \lookahead\lceil\dots) \dots r_n, w_1\dots\lceil \mathop{w_j}^{\mathop{k}^\rightarrow}\dots w_m ) $$
\item \underline{lookahead -- koniec, skok a vymazanie záznamu}
\\ Nech ) patrí ku $k$-temu pozitívnemu lookaheadu v poradí: 
\\ $\displaystyle{(r_1\dots \lookahead\dots\lceil) \dots r_n, w_1\dots \mathop{w_l}^{\mathop{k}^\rightarrow}\dots\lceil w_j\dots w_m )} $
$$ \vdash(r_1\dots \lookahead\dots)\lceil \dots r_n, w_1\dots \lceil w_l \dots w_j\dots w_m ) $$
\item \underline{lookbehind -- začiatok, jeho záznam a skok}
\\ Nech $\lookbehind\dots)$ je $k$-ty pozitívny lookbehind v poradí, $\forall L\in\lbrace 0,\dots, j-1\rbrace$:
\\ $\displaystyle{(r_1\dots \lceil \lookbehind\dots) \dots r_n, w_1\dots\lceil w_j\dots w_m )}$
$$ \vdash (r_1\dots \lookbehind\lceil\dots) \dots r_n, w_1\dots\lceil w_{j-L}\dots \mathop{w_j}^{\mathop{k}^\leftarrow} \dots w_m ) $$
\item \underline{lookbehind -- koniec a vymazanie záznamu}
\\ Nech ) patrí ku $k$-temu pozitívnemu lookbehindu v poradí:
\\ $\displaystyle{(r_1\dots \lookbehind\dots\lceil) \dots r_n, w_1\dots \lceil \mathop{w_j}^{\mathop{k}^\leftarrow} \dots w_m )}$
$$ \vdash(r_1\dots \lookbehind\dots)\lceil \dots r_n, w_1 \dots \lceil w_j \dots w_m ) $$
\item \underline{negatívny lookahead}
\\ Ak $\nexists p \in \lbrace j,\dots,m\rbrace: (\lceil r_k\dots r_l,\lceil w_j\dots w_p) \vdash^* (r_k\dots r_l\lceil, w_j\dots w_p \lceil)$, potom:
\\$\displaystyle{(r_1\dots\lceil \nlookahead r_{k}\dots r_{l} ) \dots r_n, w_1\dots \lceil w_j \dots w_m)}$
$$\vdash (r_1\dots \nlookahead r_{k}\dots r_{l} ) \lceil\dots r_n, w_1\dots \lceil w_j \dots w_m)$$
\item \underline{negatívny lookbehind}
\\ Ak $\nexists p \in \lbrace 1,\dots,j-1\rbrace: (\lceil r_k\dots r_l,\lceil w_p\dots w_{j-1}) \vdash^* (r_k\dots r_l\lceil, w_p \dots w_{j-1} \lceil)$, potom:
\\$\displaystyle{(r_1\dots\lceil \nlookbehind r_{k}\dots r_{l} ) \dots r_n, w_1\dots \lceil w_j \dots w_m)}$
$$\vdash (r_1\dots \nlookbehind r_{k}\dots r_{l} ) \lceil\dots r_n, w_1\dots \lceil w_j \dots w_m)$$
\end{enumerate}
%%%%%%%%%%%%%%%%%%%%%%%%%%%%%%%%%%%%%
\end{definition}

\textbf{Akceptačný výpočet} je postupnosť konfigurácií $(\lceil R, \lceil W) \vdash^* (R\lceil,w\lceil)$. Ak existuje akceptačný výpočet pre daný regex $R$ a slovo $W$ hovoríme, že regex $R$ matchuje slovo $W$ respektívne slovo $W$ vyhovuje regexu $R$. \textbf{Jazyk} vyhovujúci danému regexu je množina slov, pre ktoré existuje akceptačný výpočet.
\\ \\ Vďaka týmto definíciám sme schopný odhadnúť dĺžku výpočtu:
\begin{lemma}
Nech $\alpha \in \el$ a $w \in L(\alpha)$. Potom existuje akceptačný výpočet, ktorý má najviac $5\cdot|\alpha|^2\cdot|w|$ konfigurácií.
\end{lemma}
\begin{proof}
Vo väčšine krokov výpočtu sa posúvame dopredu buď v regexe alebo v slove alebo v oboch. Takéto kroky vedú k postupnosti dĺžky najviac $|\alpha|+|w|$.

V krokoch VIII. a X. sa žiaden ukazovateľ nepohne. Oba kroky sa vyskytujú 1x ku každej spätnej referencii, ktorých je najviac $|\alpha|$, takže tieto konfigurácie sa objavia najviac $2\cdot|\alpha|$-krát.

Výpočet môže predĺžiť skákanie ukazovateľa dozadu. V krokoch VI.(2),VII.(2) sa na Kleeneho $*$ rozhodneme urobiť ďalšiu iteráciu. Zamyslime sa nad samotným akceptačným výpočtom. Ak existuje, potom existuje aj taká jeho verzia, kde každé opakovanie regexu pomocou Kleeneho $*$ matchuje aspoň 1 znak -- prázdne iterácie môžeme vyhodiť, lebo ukazovateľ v slove zostal na mieste a konfigurácia skoku je rovnaká, takže postupnosť sa priamo naviaže. Vieme, že regex opakovaný operáciou $*$ je dlhý nanajvýš $|\alpha|$ znakov a podľa úvahy ho treba opakovať najviac $|w|$-krát.

Teda dokopy spravíme najviac $|\alpha|+|w|+2\cdot|\alpha|+|\alpha|\cdot|w| \leq 5\cdot|\alpha|\cdot|w|$ krokov.
\end{proof}

\begin{lemma}\label{lema_dlzka_post_konfig}
Nech $\alpha \in \lel,~ w \in L(\alpha),~ r = |\alpha|$ a $v = |w|$. Potom existuje akceptačný výpočet, ktorý má nanajvýš $O(r^2w^3)$ konfigurácií.
\end{lemma}
\begin{proof}
Spočítajme, koľko je všetkých konfigurácií pre regex $\alpha$ a slovo $w$.

Ukazovateľ v regexe môže mať $(r+1)$ rôznych pozícií. Ukazovateľ v slove môže mať $(v+1)$ rôznych pozícií. V slove môžu byť niekedy 2 ukazovatele. Spolu máme $(r+1)(w+1)+(r+1)(w+1)^2 = (r+1)(w+1)(w+2)$ možností.

V konfiguráciách sú v poschodových symboloch zapísané informácie potrebné k výpočtu. Pre každé spätné referencie sú to 2 zápisy a pre každý lookaround 1. Dokopy je to najviac $2r$ zápisov. Každá informácia buď v regexe ešte nie je zapísaná alebo má $w$ možností, kde zapísaná môže byť. Pre informácie máme dokopy $2r(w+1)$ možností.

Všetkých možných konfigurácií je dohromady
\begin{equation}\label{eq_pocet_konfig}
(r+1)(w+1)(w+2)\cdot 2r(w+1) = O(r^2w^3)
\end{equation}

Späť k akceptačnému výpočtu. Nech sú všetky akceptačné výpočty dlhšie ako \eqref{eq_pocet_konfig}. Potom tento výpočet musí nutne obsahovať 2 rovnaké konfigurácie. Keď vymažeme úsek medzi rovnakými konfiguráciami a napojíme postupnosť, dostaneme opäť validný akceptačný výpočet. Postup opakujeme, pokým výpočet obsahuje nejaké 2 rovnaké konfigurácie. Keď skončíme, sú všetky konfigurácie vo výpočte rôzne a preto má najviac \eqref{eq_pocet_konfig} konfigurácií.
\end{proof}

\section{Vlastnosti lookaroundu}
Na zoznámenie s novou operáciou sme skúsili zistiť uzavretosť tried Chomského hierarchie.
\begin{theorem}
$\R$ je uzavretá na negatívny a pozitívny lookaround.
\end{theorem}
\begin{proof}
Nech $L_1,L_2,L_3\in\R$. Chceme ukázať, že $L_1\lookahead L_2)L_3,~ L_1\lookbehind L_2)L_3,~ L_1\nlookahead L_2)L_3,~ L_1\nlookbehind L_2)L_3 \in \R$. Pre každé $L_i$, $i \in \lbrace 1,2,3\rbrace$ existuje determinický konečný automat $A_i$, ktorý ho akceptuje.

Konečné automaty vieme vhodne pospájať dohromady. Spravíme konštrukciu pre prienik regulárnych jazykov, ale mierne upravenú tak, že akonáhle automat pre $L_2$ v pozitívnom lookaheade akceptuje, vo výpočte bude pokračovať už len samotný $A_3$, kým dočíta slovo. Podobne pre pozitívny lookbehind -- $A_1$ začne sám a nedeterministicky v nejakom kroku začne výpočet aj $A_2$. Akceptovať musia spolu.

Pre negatívne formy musíme navyše upraviť akceptáciu. Ak $A_2$ pre lookahead akceptuje, celý automat sa zasekne a zamietne. $A_2$ musí dočítať slovo bez dosiahnutia akceptačného stavu. Pre lookbehind v každom kroku $A_1$ spúšťame ďalší $A_2$ a držíme si množinu stavov, v ktorých sa všetky nachádzajú. Úspech je, ak $A_1$ akceptuje a množina stavov pre automaty $A_2$ neobsahuje akceptačný stav.

Odhliadnuc teraz od lookaroundu, máme zreťazenie $L_1$ a $L_3$. Preto prepojíme akceptačný stav $A_1$ s počiatočným stavom $A_3$. Výsledný automat sa nedeterministicky rozhoduje, či z akceptačného stavu $A_1$ pokračuje ďalej v $A_1$ alebo $A_3$.
\end{proof}

\begin{theorem}
$\L_{CF}$ nie je uzavretá na pozitívny lookaround.
\end{theorem}
Vieme totiž vygenerovať jazyk $L=\lbrace a^nb^nc^n|n\in\N\rbrace$ pomocou jazykov $L_1=\lbrace a*b^nc^n|n\in\N\rbrace$ a $L_2=\lbrace a^nb^nc*|n\in\N\rbrace$: $L=\lookahead L_1)L_2 =L_1\lookbehind L_2)$.

Takýto výsledok bol očakávateľný, pretože lookaround robí uzavretosť na prienik a to trieda bezkontextových jazykov nie je.

\begin{theorem}
$\L_{CS}$ je uzavretá na pozitívny lookaround.
\end{theorem}
\begin{proof}
Nech jazyky $L_1,L_2,L_3\in\L_{CS}$, potom ukážeme, že $L_1\lookahead L_2)L_3, ~L_1\lookbehind L_2) L_3 \in\L_{CS}$. Ku každému $L_i$ existuje Turingov stroj $T_i$, ktorý ho akcpetuje. Zostrojíme nedeterministický Turingov stroj $T$ pre lookahead.

$T$ bude mať vstupnú read--only pásku. $T$ si nedeterministicky rozdelí vstupné slovo $w$ na $w_1,w_2, w_3$ tak, že $w = w_1w_3$ a $w_3=w_2x$ pre nejaké $x$ ($w_1=xw_2$ v prípade lookbehindu). Na jednu pracovnú pásku prepíše $w_1$ a bude simulovať $T_1$. Keď akceptuje, pásku vymaže, zapíše tam $w_2$ a bude simulovať $T_2$. Ak aj ten akceptuje, pásku vymaže, zapíše tam $w_3$ a bude simulovať $T_3$. Ak $T_3$ akceptuje, bude akceptovať aj $T$. Zrejme akceptuje požadovaný jazyk.
\end{proof}

\begin{theorem}
Trieda jazykov nad $\re$ s pozitívnym a negatívnym lookaroundom je ekvivalentná $\R$.
\end{theorem}
\TODO

\section{Chomského hierarchia}
\begin{theorem}
$\R \subsetneq \el \subsetneq \lel \subseteq \nlel \subsetneq \L_{CS}$
\end{theorem}
\begin{proof}

Všetky $\subseteq$ triviálne platia.

$\R \subsetneq \el$: \cite{ExtendedRegexPower}

$\el \subsetneq \lel$: Nerovnosť dokazuje jazyk $L=\lbrace a^iba^{i+1}ba^ik ~|~ k=i(i+1)k' \text{ pre nejaké } k'>0,i>0\rbrace$. $L\notin\e$ podľa pumpovacej lemy z \cite{ExtendedRegexIntersec} a tu je regex z $\le$ pre L: 
$\alpha=\displaystyle{\mathop{(}_1 a*\mathop{)}_1 b \mathop{(}_2 \backslash 1 a \mathop{)}_2 b \lookahead (\backslash 1) *\$ ) (\backslash 2)* }$

$\lel,\nlel \subsetneq \L_{CS}$: Triedy $\lel$ a $\nlel$ sú neporovnateľné s $\L_{CF}$. K jazyku $L_1=\lbrace ww~|~w\in \lbrace a,b \rbrace^*\rbrace \notin \L_{CF}$ existuje regex z $\le$: $\alpha= ((a|b)*)\backslash 1$. Ani jedna z tried neobsahuje jazyk $L_2=\lbrace a^nb^n|n\in\N\rbrace \in \L_{CF}$.
\end{proof}

Intuitívne by malo platiť aj $\lel \subsetneq \nlel$, pretože negatívny lookaround pridáva uzavretosť na komplement. Jazyk dokazujúci nerovnosť by mohol byť napríklad regex $\alpha =\nlookahead ~(aa+)(\backslash 1)+ \$~)$, pričom $L(\alpha)= \lbrace a^p ~|~ p \text{ je prvočíslo} \rbrace$. Avšak na dokázanie $L(\alpha)\notin\lel$ zatiaľ nemáme šikovné prostriedky a preto to zostáva netriviálnym otvoreným problémom.

\section{Vlastnosti triedy $\lel$}

Očividne operácia lookahead/lookbehind pridala uzavretosť na prienik. Nech $\alpha,\beta\in\le$, potom $L(\alpha)\cap L(\beta) = L(\gamma)$, kde $\gamma = \lookahead \alpha \$)\beta$ alebo $\beta \lookbehind ~\textasciicircum \alpha)$.

Napak ohrozila uzavretosť na základnú operáciu -- zreťazenie. Pri zreťazení 2 jazykov, ktorých regexy nutne musia obsahovať lookahead resp. lookbehind nastáva problém. Nemôžeme tieto regexy len tak položiť za seba. Ak sa napríklad v prvom z jazykov nachádza lookahead, počas výpočtu môže zasahovať aj do časti vstupu, ktorú matchuje druhý regex a tým zmeniť výsledok celého výpočtu. Nakoniec sa ukázalo:
\begin{theorem}
$\lel$ je uzavretá na zreťazenie.
\end{theorem}
\begin{proof}
Nech $\alpha,\beta\in\le$. Jazyku $L(\alpha)L(\beta)$ bude zodpovedať regex
$$
\gamma = \lookahead \mathop(_1 \alpha \mathop)_1 \mathop(_{k+2}\beta \mathop)_{k+2} \mathdollar) ~\alpha ' \backslash k\text{+2}~ \lookbehind ~\textasciicircum\backslash 1~\beta ')
$$
V $\alpha,\beta$ treba vhodne prepísať označenie zátvoriek (po poradí). $\alpha '$ je $\alpha$ prepísaný tak, že pre každý lookahead:
\begin{itemize}
\item bez \$ -- na koniec pridáme $.* \backslash k+2 \mathdollar $
\item s \$ -- pred \$ pridáme $\backslash k+2$
\end{itemize}
$\beta '$ je $\beta$ prepísaný tak, že pre každý lookbehind:
\begin{itemize}
\item bez $\textasciicircum$ -- na začiatok pridáme $\textasciicircum \backslash 1.*$
\item s $\textasciicircum$ -- pred $\textasciicircum$ pridáme $\textasciicircum \backslash 1$
\end{itemize}

Slovami vyjadrené, regex $\gamma$ najprv rozdelí vstupné slovo na 2 podslová $w_1,w_2$ patriace do príslušných jazykov $L(\alpha), L(\beta)$. Potom spustí ešte raz regex $\alpha$ upravený tak, že jeho lookaheady sú ,,skrotené'', pretože ich na konci donúti matchovať $w_2$. Rovnako lookbehindy v $\beta'$ donúti na začiatku matchovať $w_1$, až potom normálne pokračuje ich výpočet.

Zrejme $L(\gamma)=L(\alpha)L(\beta)$.
\end{proof}

\begin{theorem}
Nech $\alpha\in\le$ nad unárnou abecedou $\Sigma = \lbrace a \rbrace$, že neobsahuje lookahead s $\mathdollar$ ani lookbehind s $\textasciicircum$ vnútri iterácie. Existuje konštanta $N$ taká, že ak $w \in L(\alpha)$ a $\vert w \vert > N$, potom existuje dekompozícia $w=xy$ s nasledujúcimi vlastnosťami:
\begin{enumerate}[(i)]
\item $\vert y \vert \geq 1$
\item $\exists k \in \N,~k\neq 0;~\forall j = 1,2,\ldots: xy^{kj} \in L(\alpha)$
\end{enumerate}
\end{theorem}
\begin{proof}
\TODO
\end{proof}

\begin{theorem}
Jazyk všetkých platných výpočtov Turingovho stroja patrí do $\lel$.
\end{theorem}
\begin{proof}
Takýto jazyk pre konkrétny Turingov stroj $M$ obsahuje slová, ktoré sú tvorené postupnosťou konfigurácií oddelených oddeľovačom \#. Každá postupnosť zodpovedá akceptačnému výpočtu na nejakom slove. Jazyk obsahuje akceptačné výpočty na všetkých slovách, ktoré sú v jazyku $L(M)$.

Turingov stroj má konečný zápis, preto je možné regex pre takýto jazyk vytvoriť. Konštrukcia regexu: $\alpha = \beta(\gamma)*\eta$, kde $\beta$ predstavuje počiatočnú konfiguráciu\footnote{Musí byť previazaná s nasledujúcou konfiguráciou, aby spĺňala $\delta$-funkciu. Spraví sa to pomocou lookaheadu, podobne ako v $\gamma_1$.} a $\eta$ akceptačnú konfiguráciu. Ak $q_0$ je akceptačný stav, potom na koniec $\alpha$ pridáme $|(\#q_0.*\#)$. $\gamma = \gamma_1~|~\gamma_2~|~\gamma_3$.	Prvok $\gamma_i$ generuje validnú konfiguráciu a zároveň kontroluje pomocou lookaheadu, či nasledujúca konfigurácia môže podľa $\delta$-funkcie nasledovať. Rozpíšeme si iba jednu možnosť: 
 $$ \displaystyle \gamma_1=(\mathop(_k.*\mathop)_k x q y \mathop(_{k+1} .* \mathop)_{k+1} \#)(?= \xi \#)$$ platí pre $\forall q \in K,~\forall y \in \Sigma$ a kde $ \xi = \xi_1 ~|~ \xi_2 ~|~ \dots ~|~ \xi_n $.
\begin{itemize}
\item Ak $ (p,z,0) \in \delta(q,y)$, potom $\xi_i = (\backslash k~x p z \backslash k+1)$ pre nejaké $i$
\item Ak $ (p,z,1) \in \delta(q,y)$, potom $\xi_i = (\backslash k~x z p \backslash k+1)$ pre nejaké $i$
\item Ak $ (p,z,-1) \in \delta(q,y)$, potom $\xi_i = (\backslash k~p x z \backslash k+1)$ pre nejaké $i$
\end{itemize}

$\gamma_2$ a $\gamma_3$ sú podobné ako $\gamma_1$, ale matchujú krajné prípady, kedy je hlava Turingovho stroja na ľavom alebo pravom konci pásky.

Zrejme $L(\gamma)$ je požadovaný jazyk.
\end{proof}

\section{Priestorová zložitosť}
\begin{theorem}\label{lere_nspace(logn)}
$\lel \subseteq NSPACE(\log n)$, kde $n$ je veľkosť vstupu.
\end{theorem}
\begin{proof}
Nech $\alpha\in\le$. Zostrojíme nedeterministický Turingov stroj T akceptujúci $L(\alpha)$, ktorý bude mať vstupnú read--only pásku a 1 jednosmerne nekonečnú pracovnú pásku, na ktorej zapíše najviac $\log n$ políčok.

Výpočet Turingovho stroja bude prebiehať podľa postupnosti konfigurácií formálneho modelu. Nemôžeme nič zapísať na vstupnú pásku a máme k dispozícii menej priestoru ako je dĺžka vstupu. Využijeme to, že pre vstup dĺžky $n$ vieme uložiť ľuboboľnú pozíciu na vstupe do adresy dĺžky $\log n$. Ukážeme, že takýchto adries potrebujeme konečný počet. Potom ich vieme písať nad seba do niekoľkých stôp pásky a mať tak zapísaných najviac $\log n$ políčok na páske.

Celý regex $\alpha$ bude uložený v stave aj s ukazovateľom. Budú existovať stavy pre všetky možné pozície ukazovateľa v regexe a medzi stavmi budú tzv. metaprechody podľa definície kroku výpočtu na regexe. Medzi každými dvoma stavmi prepojenými metaprechodom môže byť potrebných až niekoľko prechodov cez pomocné stavy (napríklad keď narazí na otváraciu indexovateľnú zátvorku, na ktorú sa odkazujú spätné referencie, musí zapísať aktuálnu pracovnú adresu v slove ako začiatok podslova).

Adresy budú zaznamenávať všetky ostatné informácie v konfigurácii -- aktuálnu pracovnú pozíciu na vstupe (ukazovateľ v slove), pomocný ukazovateľ na spätné referencie, začiatok a koniec podslova zodpovedajúceho $k$-tym zátvorkám pre $\forall k$ (počet $z$), začiatok každého lookaheadu ($l_a$) a look\-be\-hindu ($l_b$). K tomu bude potrebná 1 pomocná adresa -- aktuálna pozícia hlavy na vstupe. Z definície $\alpha$ je konečnej dĺžky a pre počty daných operácií platí $2z+l_a+l_b \leq |\alpha|$. Spolu máme $2+2z+l_a+l_b+1 \leq |\alpha|+3$, čo je konštanta. 

Kroky výpočtu I., IV., V., VI., VII.(1) nepotrebujú pomocné stavy. Ostatné kroky zapisujú, prepisujú a porovnávajú adresy. Zápis aktuálnej adresy ( je len kopírovanie znakov z inej stopy), vynulovanie záznamu a porovnávanie niekoľkých stôp vyžaduje 1 prechod cez pracovnú pásku a žiadnu prídavnú pamäť.

Preto T akceptuje $L(\alpha)$ a spĺňa pamäťové požiadavky.
\end{proof}

Dôsledok Savitchovej vety \cite{Savitch}:
\begin{theorem}\label{lel_dspace}
$\lel \subseteq DSPACE(\log^2 n)$, kde $n$ je veľkosť vstupu.
\end{theorem}
\begin{theorem}\label{nlel_dspace}
$\nlel \subseteq DSPACE(\log^2 n)$, kde $n$ je veľkosť vstupu.
\end{theorem}
Dôkaz tejto vety uvedieme neskôr.

V praxi je bežné, že užívateľ zadáva nielen vstupný text, ale aj samotný regex. Preto sme sa rozhodli analyzovať jazyk, ktorý dostane na vstup oboje -- slovo $regex\#word$-- a akceptuje slovo len vtedy, ak slovo $word$ vyhovuje regexu $regex$.

\begin{theorem}
$L(regex\#word) \in NSPACE(r \log w)$, kde $r = |regex|,~w = |word|$ a $regex \in \le$.
\end{theorem}
\begin{proof}
Myšlienka dôkazu je podobná ako v dôkaze \ref{lere_nspace(logn)}. Rozdiel je v tom, že regex nepoznáme dopredu. Z čoho vyplýva, že si ho nemôžeme uchovať v stave. Preto pribudnú ďalšie 2 adresy -- pracovná pozícia v regexe (ukazovateľ) a aktuálna pozícia v regexe. Ďalším dôsledkom je, že síce počet adries ohraničíme zhora číslom $r+3$, ale už to viac nie je konštanta. Preto adresy nemôžeme ukladať na viacerých stopách pod sebou, ale musia byť vedľa seba oddelené oddeľovačmi. Pre rovnako pohodlné porovnávanie a zapisovanie si môžeme dovoliť pridať 1 pracovnú pásku, na ktorú si 1 z porovnávaných adries zapíšeme -- tá bude mať vždy najviac $\log w$ zapísaných políčok.

Turingov stroj bude fungovať ako v dôkaze \ref{lere_nspace(logn)}, ale odhad zapísanej pamäte bude $(r+3)\cdot \log w + 2\log r$. Všetky pozície v slove vieme adresovať od oddeľovača \#, preto zaberú $\log w$ pamäte. Na záver pribudla pracovná a aktuálna pozícia v regexe, z nich každá potrebuje $\log r$ políčok. Dokopy Turingov stroj zapíše $O(r\log w)$ pamäte.
\end{proof}

\begin{theorem}\label{regexword_dspace}
$L(regex\#word) \in DSPACE(n \log^2 n)$, kde $regex \in \le$ a $n$ je dĺžka vstupu.
\end{theorem}
\begin{proof} 
Nech $r=|regex|,~w=|word|$.

Myšlienka je podobná dôkazu Savitchovej vety\cite{Savitch_skripta}. Turingov stroj T bude testovať, či sa dá dostať z konfigurácie $C_1$ do konficurácie $C_2$ na $i$ krokov:
\lstset{numbers=left, tabsize=4, morekeywords={if,then,return,true,false}, columns=fullflexible, breaklines=true, numbersep=2pt,backgroundcolor=\color{ltyelloow}}
\begin{lstlisting}[mathescape]
bool $TESTUJ(C_1,C_2,i)$
	if ($C_1 == C_2$) then return true
	if ($i>0 ~\wedge~ C_1\vdash C_2$) then return true
	if ($i<=1$) return false
	$\text{iteruj cez všetky konfigurácie } C_3$
		if $(TESTUJ(C_1, C_3,\lfloor \frac{i}{2}\rfloor)$ $\wedge ~TESTUJ(C_3, C_2,\lceil \frac{i}{2}\rceil))$ then 		return true
	return false
\end{lstlisting}
Konfigurácie budú zodpovedať formálnemu modelu a ako v predošlom dôkaze budú na páske zaznamenané ako niekoľko adries -- pracovná pozícia v regexe, pracovná pozícia v slove, začiatok a koniec podslova pre $k$-te indexovateľné zátvorky pre $\forall k$ $(z)$, začiatok každého lookaheadu $(l_a)$ a lookbehindu $(l_b)$. Globálne si budeme pamätať ešte aktuálnu pozíciu v regexe a v slove, kvôli orientácii a prípadnému kopírovaniu adries. Spolu to zaberie $\log r+(1+2z+l_a+l_b)\cdot \log w + \log r+\log w \leq O(r\log w)$ pamäte.

Turingov stroj T začne volaním inštancie $TESTUJ(C_0,C_a,c)$, kde $C_0$ je počiatočná konfigurácia, $C_a$ je akceptačná konfigurácia a $c$ je číslo z lemy \ref{lema_dlzka_post_konfig}. Ak akceptačný výpočet existuje, potom existuje aj taký, ktorý má nanajvýš $c$ konfigurácií.

Procedúra $TESTUJ$ je rekurzívna. Preto bude na pracovnej páske stroja T zásobník. Pre každú inštanciu procedúry bude mať uložené konfigurácie $C_1, C_2, C_3, c$ a informáciu, či sa vrátil z prvého alebo druhého volania (potrebný 1 bit informácie). Hodnotu $c$ vieme zapísať do priestoru $\log c = O(\log r + \log w)$, teda jeden záznam tak zaberie $3r\log w + \log c = O(r\log w)$ pamäte. Keďže parameter $i$ je vždy o polovicu menší, hĺbka rekurzie bude $\log c$.

Z toho vyplýva, že zásobník bude potrebovať $O((\log r + \log w)\cdot r\cdot \log w) = O(n\log^2n)$ pamäte. Ešte treba overiť, že úkony na riadkoch 2--4 zvládne T vykonať tiež v rámci pamäťového limitu.

Riadok 2 je porovnanie rovnosti adries -- pracovnú pozíciu porovnávania si môže značiť poschodovými symbolmi. Riadok 4 je triviálny. Riadok 3 je zložitý kvôli overeniu $C_1\vdash C_2$. K tomu potrebuje nasledovné kontroly:
\begin{description}
\item[ukazovateľ] -- či je správne posunutý ukazovateľ (týka sa aj špeciálneho, ak je aktívny). To znamená, že buď má byť posunutý o konkrétny počet políčok alebo má byť vľavo/vpravo a ukazovať na konkrétny symbol.
\item[adresy] -- všetky adresy (mimo ukazovateľov) musia byť rovnaké, okrem tých, ktorým je v tomto kroku nastavená nová hodnota. Tá musí byť korektne nastavená (t.j. rovnaká ako ukazovateľ v slove). 
\item[zátvorky] -- pre korektné skoky v regexe v krokoch II.(2) a VII.(2) musí byť medzi starou a novou pozíciou ukazovateľa počet otváracích a zatváracích zátvorkiek rovnaký.
\item[alternovateľnosť] -- pokiaľ sa jedná o skok v alternácii (IV.,V.), treba skontrolovať prvý alebo posledný alternovateľný regex.
\item[indexovateľnosť] -- ak zátvorka nie je indexovateľná, tak sme narazili na lookaround.
\end{description}
Indexovateľnosť a ukazovateľ sa skontrolujú bez použitia pomocnej pamäte. Adresy využívajú porovnávanie, ale to vieme spraviť pomocou poschodových symbolov. Alternovateľnosť využíva algoritmus na kontrolu zátvoriek -- zisťuje, či je alternácia uzavretá zátvorkami (ak hej, ktorými) alebo nie. Počet zátvoriek je najviac $\frac{r}{2}$. Používame algoritmus, kde je ( priradíme 1 a ) hodnotu -1\footnote{Ak počítame sprava doľava, obe hodnoty prenásobíme $(-1)$, aby sme pri prvej zátvorke nemali súčet $0+(-1)$.}. Pri každom výskyte sa hodnoty sčítavajú, 0 je dobre uzátvorkovaný výraz. Kontrola sa vykoná a súčet po nej už nepotrebujeme, preto ho môžeme dočasne zapísať na koniec zásobníka a vzápätí vymazať. Zapísaná pamäť tak bude $r+O(n\log^2n)=O(n\log^2n)$.
\end{proof}

Tu už nasleduje sľubovaný dôkaz vety \ref{nlel_dspace}. Budeme čerpať z dôkazu predošlej vety.
\begin{proof}
Nech $\alpha\in\nle$ a $r=|\alpha|$. Zostrojíme Turingov stroj $T$, ktorý bude akceptovať $L(\alpha)$ a na pracovných páskach nezapíše viac ako $O(\log^2 n)$ políčok.

Pokiaľ $\alpha$ neobsahuje negatívny lookaround, tvrdenie triviálne vyplýva z vety \ref{lel_dspace}. Nech teda obsahuje aspoň jeden negatívny lookaround a nech $k$ je najvyšší počet negatívnych lookaroundov vnorených do seba (nezáleží na tom, či sú to lookaheady alebo look\-behindy).

$T$ bude skonštruovaný ako Turingov stroj vo vete \ref{regexword_dspace}, pričom budeme musieť dodefinovať správanie v prípade negatívneho lookaroundu. V definícii \ref{dfkonfig} v bodoch XV. a XIV. je napísané, že ak splníme nejakú podmieku, negatívny lookaround možno preskočiť a pokračovať ďalej vo výpočte. Podmienka začína ,,neexistuje výpočet'', čo naznačuje, že musíme vyskúšať všetky možnosti -- teda mať deterministický algoritmus, ktorý akceptuje práve vtedy, keď akceptačný výpočet existuje.

Vhodným algoritmom je procedúra $TESTUJ$. Zakaždým, keď $T$ bude overovať podmienku $C_1\mathop{\vdash}^? C_2$ a jedná sa o prechod XV. alebo XIV. z definície \ref{dfkonfig}, stane sa nasledovné. $T$ spustí na novej páske novú procedúru $TESTUJ(C_0',C_a',c')$, kde $C_0'$ je počiatočná a $C_a'$ akceptačná konfigurácia z definície daného kroku a $c' \leq c$ je hodnota z lemy \ref{lema_dlzka_post_konfig} pre regex vo vnútri tohto negatívneho lookaroundu. Túto procedúru treba spustiť niekoľkokrát po sebe -- pre každé $p$, čo prichádza do úvahy. Pokiaľ niektorý z behov procedúry $TESTUJ$ skončí úspešne, znamená to, že existuje akceptačný výpočet tam, kde nechceme, aby existoval -- podmienka negatívneho lookaroundu neplatí a výsledok je $C_1\nvdash C_2$. Ak všetky behy skončia s výsledkom \textit{false}, výsledok je $C_1\vdash C_2$.

Vynechali sme detail ,,spustenie pre každé p, čo prichádza do úvahy''. Tu je treba zadať hranice podslova, na ktorom sa pracuje, a neprekročiť ich. Jednoducho zakomponujeme do procedúry kontrolu, či sú všetky adresy a ukazovateľ pre toto spustenie povolené. Tieto hodnoty sú globálne a zapisujú sa pri prvom volaní na začiatok zásobníka. Pre hlavný beh procedúry to budú hodnoty 0 a $n$.

Popísali sme správanie $T$, pre prípady, keď operácie negatívneho lookaroundu nie sú vnorené. Opíšme tie. $T$ má na 1. páske rozpracovanú hlavnú vetvu $TESTUJ$, teraz pracuje na 2. páske na negatívnom lookarounde a narazí na ďalší. Uvedomme si, že pre $T$ je to rovnaká situácia, akokeby pracoval stále na 1. páske. Zopakuje postup popísaný vyššie -- niekoľkokrát spustí $TESTUJ$ na 3. páske pre vhodné hranice slov a ak výsledkom kadžého behu bude \textit{false}, vráti sa na 2. pásku. Nech regex $\alpha$ má $k$ vnorených negatývnych lookaroundov, potom $T$ bude potrebovať $k+1$ pracovných pások.

Pre spočítanie zapísaných políčok si najprv popíšme konfigurácie. Regex poznáme dopredu. To znamená, že pre každú polohu ukazovateľa v regexe vieme mať znak v pracovnej abecede (t.j. $r+1$ špeciálnych symbolov). Zároveň pre výpočty na negatívnych lookaroundoch nám stačí ich vnútorný regex s ukazovateľom. Takýchto podslov je konečne veľa, preto aj pre tie vieme mať samostatné špeciálne symboly. 

Adresy, ktoré v konfiguráciách potrebujeme sú: pracovná pozícia v slove, začiatok a koniec podslova pre $k$-te zátvorky pre $\forall k$ $(z)$, začiatok každého lookaheadu $l_a$ a lookbehindu $(l_b)$. Spolu $1+2z+l_a+l_b \leq 2r+1$ adries a to je konštanta. Konštantný počet adries vieme umiestniť nad seba do konštantného počtu stôp na páske (ako v dôkaze \ref{lere_nspace(logn)}) a takto nimi zaberieme $\log n$ políčok (symbol pre stav regexu bude v samostatnej najvrchnejšej stope).

Jedno volanie procedúry $TESTUJ$ potrebuje 3 konfigurácie a konštantný počet políčok. Hĺbka vnorenia rekurzie je na každej páske najviac $\log c=O(\log r + \log w) = O(\log n)$. Dokopy bude na každej páske zapísaných najviac $\log n \cdot O(\log n) = O(\log^2n)$ políčok.
\end{proof}


\section*{Poďakovanie}
Ďakujem školiteľovi za cenné rady a pripomienky.


\nocite{*}
\bibliographystyle{apalike}
\bibliography{references}

\end{document}

