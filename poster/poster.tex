% !Mode:: "TeX::UTF-8"
\documentclass[myposter,portrait,plainboxedsections]{sciposter}

%% uzitocne package
\usepackage{multicol}
\usepackage{xcolor}
\usepackage{graphicx}

%% znaky s diakritikou
\usepackage[utf8]{inputenc}
\usepackage[T1]{fontenc}
\usepackage{fontenc}
\usepackage[slovak]{babel} % slovenske delenie slov

% matika
\usepackage{amssymb}
\usepackage{amsmath}
\usepackage{amsthm}
\usepackage{dsfont}
\usepackage{mathrsfs}
\usepackage{amsfonts}
\usepackage{lmodern}
\usepackage{bm} %pisat boldom v math-mode

% Vennove diagramy
\usepackage{tikz}
\usetikzlibrary{shapes,positioning,fit}

%% definicia farieb
\definecolor{mainCol}{rgb}{1,1,0.88} % farba pozadia posteru
\definecolor{sectionCol}{rgb}{1,1,0} % farba ramceka boxov
\definecolor{textCol}{rgb}{0,0,0} % farba nadpisu 
\definecolor{BoxCol}{rgb}{1,1,0.2} % farba boxu okolo nadpisov

% silna zlta
\definecolor{cadmiumyellow}{rgb}{1.0, 0.96, 0.0}
\definecolor{electricyellow}{rgb}{1.0, 1.0, 0.0}
\definecolor{laserlemon}{rgb}{1.0, 1.0, 0.13}
\definecolor{yellow}{rgb}{1.0, 1.0, 0.0}
% slaba zlta
\definecolor{icterine}{rgb}{0.99, 0.97, 0.37}
\definecolor{unmellowyellow}{rgb}{1.0, 1.0, 0.4}

\definecolor{lemonchiffon}{rgb}{1.0, 0.98, 0.8}
\definecolor{lightgoldenrodyellow}{rgb}{0.98, 0.98, 0.82}
\definecolor{lightyellow}{rgb}{1.0, 1.0, 0.88}
\definecolor{pastelyellow}{rgb}{0.99, 0.99, 0.59}

\def\R{{\cal R}} % znak pre regulárne jazyky
\def\L{\mathscr{L}} % zvyšné triedy jazykov
\def\P{{\cal P}} % potenčná množina
\def\N{\mathds{N}} %prirodzene cisla
\def\re{Regex}
\def\e{Eregex}
\def\le{LEregex}
\def\nle{nLEregex}
\def\rel{{\mathscr{L}_{RE}}}
\def\el{\mathscr{L}_{ERE}}
\def\lel{\mathscr{L}_{LERE}}
\def\nlel{\mathscr{L}_{nLERE}}
\def\lookahead{\text{(?=}}
\def\nlookahead{\text{(?!~}}
\def\lookbehind{\text{(?\textless =}}
\def\nlookbehind{\text{(?\textless !~}}

\def\mysection#1{
{\color{sectionCol}\section*{\sc\bfseries #1}}}

\begin{document}
\setlength{\logowidth}{20cm}
\setlength{\titlewidth}{\textwidth}
\addtolength{\titlewidth}{-\logowidth}
\rightlogo[0.9]{fmfilogo-farebne}
\useleftlogofalse

\color{textCol}

\title{Moderné regulárne výrazy}
\author{Tatiana Tóthová\\
        Školiteľ: RNDr. Michal Forišek PhD.}
\institute{%
Katedra informatiky,
FMFI UK, Mlynská Dolina, 842~48~Bratislava
}
\maketitle

\begin{multicols*}{3}

\mysection{Úvod}

Regulárne výrazy vznikli v 60tych rokoch v teórii jazykov ako ďalší model na vyjadrenie regulárnych jazykov. Z ich popisu ľudský mozog rýchlejšie pochopil o aký jazyk sa jedná, než zo zápisu konečného automatu, či regulárnej gramatiky. Ďalšou výhodou bol kratší a kompaktný zápis.

Vďaka týmto vlastnostiam boli implementované ako vyhľadávací nástroj. Postupom času sa iniciatívou používateľov s vyššími nárokmi pridávali nové konštrukcie na uľahčenie práce. Nástroj takto rozvíjali až do dnešnej podoby. My sa budeme opierať o špecifikáciu regulárnych výrazov v jazyku Python \cite{Python3Documentation}.

Nové regulárne výrazy vedia reprezentovať zložitejšie jazyky ako regulárne, preto je dobré ich odlíšiť. V literatúre sa zaužíval výraz ,,regex'' z anglického \textit{regular expression}, ktorý budeme používať aj my.

\mysection{Definícia moderných regulárnych jazykov}

\underline{Základné regulárne výrazy} sa skladajú zo znakov a metaznakov. Znak je regulárny výraz predstavujúci sám seba. Metaznak (alebo skupina metaznakov) určuje, čo sa s regulárnym výrazom udeje. Základný model obsahuje operácie
\begin{itemize}
\item \textbf{zreťazenie} ($\alpha\beta$)
\item \textbf{alternácia} ($\alpha|\beta$) 
\item \textbf{Kleeneho uzáver} ($\alpha*$ = opakuj $\alpha$ $(0-\infty)$-krát)
\end{itemize}
Okrem toho používame \textbf{zátvorky ()} na špecifikovanie poľa pôsobnosti operácií. Každý regex je zložený z konečného počtu operácií.
\\ \\
\underline{Moderné regulárne výrazy}, nazývané aj \underline{regexy}, majú navyše metaznak pre ľubovoľný znak . , začiatok slova \textasciicircum ~a koniec slova \$. Bolo zavedené číslovanie zátvoriek -- zľava doprava podľa otváracej zátvorky a konštrukcie tvaru $(?\dots)$ sa nečíslujú. Pribudli aj nasledujúce zložitejšie konštrukcie:

\begin{itemize}

%% Spätné referencie
\item \textbf{Spätné referencie} ($\backslash k$) sa odkazujú na $k$-te zátvorky (môže sa nachádzať až za nimi). Pri matchovaní slova si zapamätáme, aké podslovo matchovali $k$-te zátvorky a toto podslovo predstavuje konštrukcia $\backslash k$. Napr. pre regex
 $\displaystyle \alpha~ \mathop(_k ~\beta ~\mathop)_k ~\gamma ~\backslash k ~\delta$ by výpočet na slove $w$ vyzeral takto:

$$w = \underbrace{x_1\dots x_{i-1}}_\alpha 
 \overbrace{\underbrace{x_i\dots x_{j-1}}_{ \displaystyle{\mathop(_k\beta \mathop)_k}}}^{w_k} 
 \underbrace{x_j\dots x_{l-1}}_\gamma 
 \overbrace{\underbrace{x_l\dots x_{m-1}}_{\backslash k}}^{w_k}
 \underbrace{x_{m}\dots x_{n}}_\delta$$
a musí platiť: $w_k= x_i\dots x_{j-1} = x_l\dots x_{m-1} $. Ak existuje viac podslov ku $k$-tym zátvorkám, berie sa vždy to posledné.

%% Lookaround
\item \textbf{Lookahead} (nazeranie dopredu $\lookahead\dots)$) musí matchovať nejaký prefix od aktuálneho pracovného miesta. Napr. pre regex $\alpha(?=\beta)\gamma$ máme:
 $$w = \underbrace{x_1\dots x_{i-1}}_\alpha \underbrace{\overbrace{x_i \dots x_j}^\beta x_{j+1} \dots x_n }_\gamma$$ 

\item \textbf{Lookbehind} (nazeranie dozadu $\lookbehind\dots)$) musí matchovať nejaký sufix od aktuálneho pracovného miesta. Napr. pre regex $\alpha(?<=\beta)\gamma$ máme:

$$w = \underbrace{x_1\dots x_{i-1} \overbrace{x_i \dots x_j}^\beta}_\alpha \underbrace{x_{j+1} \dots x_n }_\gamma$$

\item Operácie lookahead a lookbehind majú aj \textbf{negatívne} verzie, kde regex vnútri konštrukcie nesmie matchovať žiaden prefix/sufix.

\end{itemize}
Pre lookahead a lookbehind sa zaužíval spoločný názov \textbf{lookaround}.
\\ \\
\underline{Zaviedli sme nasledujúce množiny operácií:}

\begin{tabular}{lll}
$\re$ ~& základné operácie &~ $\rel = \R$ \\
$\e$ ~& + spätné referencie &~ $\el$ \\
$\le$ ~& + lookahead, lookbehind &~ $\lel$ \\
$\nle$ ~& + negatívny lookahead, negatívny lookbehind &~ $\nlel$
\end{tabular}

\mysection{Formálny model} 
Moderné regulárne výrazy obsahujú množstvo operácií, ktoré spolu rôzne interagujú. Preto sme vytvorili formálny model, ktorý postupuje po krokoch podobne ako Turingov stroj. \textbf{Konfigurácia} pre regex $\alpha=r_1\dots r_n$ a slovo $w=w_1\dots w_m$ je definovaná ($\lceil$ ukazuje pracovnú pozíciu):
$$(r_1\dots \lceil r_i \dots r_n, w_1 \dots \lceil w_j \dots w_m)$$
V symboly v slove majú niekoľko poschodí, do ktorých si zapamätáme pomocnú informáciu počas výpočtu. \textbf{Krok výpočtu} je definovaný:
 $$\displaystyle (r_1 \dots \lceil ( \dots r_n, w_1 \dots \lceil w_j \dots w_m) \vdash (r_1 \dots (\lceil \dots r_n, w_1 \dots \lceil \mathop{w_j}^k \dots w_m)$$
Celú definíciu nájdete v \cite{mojaDip}.

\mysection{Vlastnosti lookaheadu a lookbehindu}

Trieda $\R$ je uzavretá na negatívny aj pozitívny lookaround. Aj trieda $\re$ s lookaroundom stále pokrýva iba regulárne jazyky -- takúto kombináciu operácií totiž vieme simulovať konečnými automatmi. Z výsledkov hierarchie tried však vyplýva nasledovné:

$$ \L \begin{pmatrix}
\re \\
\text{+lookahead} \\
\text{+lookbehind}
\end{pmatrix} 
\subsetneq
\L \begin{pmatrix}
\re \\
\text{+spätné referencie}
\end{pmatrix} 
\subsetneq
\L \begin{pmatrix}
\re \\
\text{+spätné referencie} \\
\text{+lookahead,lookbehind}
\end{pmatrix} 
$$

To znamená, že lookaround je síce sám slabá operácia, ale v kombinácii so spätnými referenciami máme silnejší model ako bez lookaroundu. A teda má zmysel vsímať si túto operáciu.
\\ \\ 
Čo sa týka \underline{uzáverových vlastností}, pozitívny lookaround pridáva uzavretosť na \textbf{prienik} $L(\alpha)\cap L(\beta) = L(~\lookahead\alpha\$)\beta~)$ a negatívny lookaround pridáva uzavretosť na \textbf{komplement} $L(\alpha)^c = L(~ \nlookahead \alpha \$).* ~)$. Avšak ohrozená je základná operácia regulárnych výrazov -- \textbf{zreťazenie}. Trieda $\lel$ však na zreťazenie uzavretá je, $L(\alpha)L(\beta)$ vyjadríme regexom:
$$
\lookahead \mathop(_1 \alpha \mathop)_1 \mathop(_{k+2}\beta \mathop)_{k+2} \mathdollar) ~\alpha ' \backslash k\text{+2}~ \lookbehind ~\textasciicircum\backslash 1~\beta ')
$$
Prvý lookahead rozdelí vstupné slovo $w$ na $w_1,w_2,~w=w_1w_2$. $\alpha'$ je regex $\alpha$ upravený tak, že jeho lookaheady na konci matchujú $w_2$ a $\beta'$ má upravené lookbehindy tak, že na začiatku matchujú $w_1$. \cite{mojaBak}
 
\mysection{Chomského hierarchia}
Triviálne platí, že v definovaných triedach je predchádzajúca podmnožinou nasledujúcej.
$$\R \mathop{\subsetneq}^{(1)} \el \mathop{\subsetneq}^{(2)} \lel \mathop{\subseteq}^{(3)} \nlel \mathop{\subsetneq}^{(4)} \L_{CS}$$
Teraz uvedieme jazyky, ktoré dokazujú nerovnosť množín. Zároveň to považujeme za malú ukážku toho, čo moderné regulárne výrazy dokážu.

(1) jazyk $L(\alpha) = \lbrace ww|w \in\lbrace a,b \rbrace^* \rbrace \in \L_{CS}$:~~~ $\alpha = (a|b)*\backslash 1$
\\ \\
(2) jazyk $L(\beta)=\lbrace a^iba^{i+1}ba^ik ~|~ k=i(i+1)k' \text{ ,kde } k'>0,i>0\rbrace \in \lel$: 
$$\beta=\mathop{(}_1 a*\mathop{)}_1 b \mathop{(}_2 \backslash 1 a \mathop{)}_2 b~ \lookahead (\backslash 1) *\$ ) (\backslash 2)* $$
\\ 
(3) nerovnosť je otvoreným problémom. Dobrým kandidátom na jej ukázanie sú jazyky:

\begin{tabular}{llc}
$L(\gamma) = \lbrace a^z | z \text{ je zložené číslo} \rbrace$ & $\in\lel$, & $\gamma = (aaa*)\backslash 1(\backslash 1)*$ \\
$L(\delta) = L(\gamma)^c = \lbrace a^p | p \text{ je prvočíslo} \rbrace$ & $\in\nlel$, & $\delta=\nlookahead \gamma\$).*$
\end{tabular}
\\ \\
(4) $\lel$ a $\nlel$ sú neporovnateľné s $\L_{CF}$

\mysection{Priestorová zložitosť}
Toto je oblasť, kvôli ktorej sme vymysleli formálny model. Všetky informácie v slove (ukazovateľ a vyššie poschodia symbolov) sa dajú zapísať vo forme adries. Adresy vieme zapísať v logaritmickom priestore od dĺžky vstupného slova. Pre regex ich potrebujeme konštantne veľa:

$$\lel \subseteq NSPACE(\log n)$$

Zo Savitchovej vety vyplýva:
$\lel \subseteq DSPACE(\log^2 n)$

Myšlienka dôkazu Savitchovej vety spočíva v tom, že testujeme, či sa vieme dostať z jednej konfigurácie do druhej na istý počet krokov. Formálny model má definované konfigurácie,ktoré sme využili a dokázali tak výsledok aj pre negatívny lookaround:
$$\nlel \subseteq DSPACE(\log^2 n)$$

V praxi je bežné, že užívateľ zadáva \underline{na vstup regex aj text na vyhľadávanie}, preto sme si zadefinovali jazyk $L_U$, ktorý akceptuje slová tvaru $regex\#word$, kde $regex\in U$, $regex$ matchuje slovo $word$ a $U$ je množina operácií.

$$L_{\le} \in NSPACE(n \log n)$$

Nech $U$ je $\e$ alebo $\le$ s konečným počtom vnorení look\-aroundov, potom:
$$L_{U} \in DSPACE(n\log ^2 n)$$

\mysection{Pojmy a skratky}
$\L(\dots)$ -- trieda jazykov nad \dots
$\R$ -- trieda regulárnych jazykov
\\ $\L_{CF}$ -- trieda bezkontextových jazykov
\\ $\L_{CS}$ -- trieda kontextovných jazykov

%% zoznam literatury
\bibliographystyle{apalike}
\nocite{*}
\bibliography{references}

\end{multicols*}
\end{document}

